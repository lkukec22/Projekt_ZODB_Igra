\documentclass[]{foi} 

\usepackage[utf8]{inputenc}
\usepackage{lipsum}
\usepackage{listings}
\usepackage{xcolor}
\usepackage{graphicx}
\usepackage{float}

% KOD STYLING
\lstset{
    language=Python,
    basicstyle=\ttfamily\footnotesize,
    commentstyle=\itshape\color{gray},
    keywordstyle=\color{blue}\bfseries,
    stringstyle=\color{red},
    showstringspaces=false,
    breaklines=true,
    prebreak=\raisebox{0ex}[0ex][0ex]{\ensuremath{\hookleftarrow}},
    numbers=left,
    numberstyle=\tiny\color{gray},
    stepnumber=1,
    backgroundcolor=\color{white},
    frame=single,
    rulecolor=\color{black},
    captionpos=b
}

\vrstaRada{\projekt}

\title{RPG Igra sa ZODB i PyGame}
\predmet{Baze Podataka}

\author{Student Studentić} 
\spolStudenta{\musko} 

\mentor{Dr. sci. Imetor Prezimento}
\spolMentora{\musko} 
\titulaProfesora{prof.~dr.~sc.}

\godina{2026}
\mjesec{siječanj}

\indeks{12345678}

\smjer{Informacijski i poslovni sustavi}


\sazetak{Ovaj projekt prikazuje razvoj RPG igre korištenjem objektno-orijentirane baze podataka ZODB i PyGame frameworka za grafiku. Aplikacija demonstrira perzistenciju kompleksnih Pythonovih objekata, upravljanje stanjem igrača, te integraciju baze podataka s grafičkim sučeljem. Igra uključuje sisteme za: kretanje igrača, sustav zdravlja, inventar stavki, te trajno pohranjevanje podataka igrača u ZODB-u. Sve operacije koriste ACID transakcije kako bi se osigurala konzistencija podataka.}

\kljucneRijeci{ZODB; objektne baze podataka; PyGame; RPG; Python; perzistencija; transakcije; okidači}

\acrodef{ZODB}{Zope Object Database}
\acrodef{RPG}{Role-Playing Game}
\acrodef{OOBP}{Objektno-orijentirana baza podataka}

\begin{document}

\maketitle

\tableofcontents

\makeatletter \def\@dotsep{4.5} \makeatother
\pagestyle{plain}

\chapter{Uvod}

Ovaj projekt je izrađen kao dio kolegija Baze podataka. Demonstrira prednosti objektnih baza podataka u razvoju igara, fokusirajući se na ACID svojstva, transakcije i transparentnu perzistenciju objekata.

Projekt je \ac{RPG} igra razvijena u 2D sa osnovnim mehanikama. Igrač kontrolira likove koji se mogu kretati, primati štetu, i skupljati predmete iz svojega inventara. Igra je dizajnirana kako bi demonstrirala kako \ac{ZODB} efektivno pohrani kompleksne objekte karakteristične za igre.

Igra posjeduje sljedeće mehanike:
\begin{enumerate}
    \item \textbf{Kretanje} -- Igrač se kreće tipkama A (lijevo) i D (desno)
    \item \textbf{Sistem Zdravlja} -- Igrač ima 100 HP, pri svakom šteti se smanjuje
    \item \textbf{Okidač za Porazu} -- Kada HP padne na 0, igrač se automatski proglašava poraženim
    \item \textbf{Inventar} -- Igrač može prikupljati stavke
    \item \textbf{Perzistencija} -- Svi podaci se trajna sprema u \ac{ZODB}
\end{enumerate}

Za ovu aplikaciju, \ac{ZODB} je odličan izbor jer omogućava da se kompleksni Python objekti (kao što je igrač s inventarom) pohranjuju direktno bez potrebe za mapiranjem na SQL tablice. Primjerice, igrač ima \texttt{PersistentList} za inventar koji sadrži \texttt{Item} objekte. Objekti se automatski prate u \ac{ZODB}-u nakon naslijeđivanja \texttt{Persistent} klase.

\chapter{Teorijski okvir}

\section{Objektno-Orijentirane Baze Podataka}

Objektno-orijentirana baza podataka (\ac{OOBP}) je sustav koji pohrani objekte direktno kao entitete baze podataka, bez potrebe za prevodom u relacijski format. Za razliku od relacijskih baza koje koriste tablice i SQL, \ac{OOBP} čuva Python objekte kako su kreirani \cite{OOBP2015}.

Osnovna svojstva usporedbe ZODB-a i relacijskih baza podataka prikazana su u tablici \ref{tab:comparison}.

\begin{table}[h!]
    \centering
    \caption{Usporedba ZODB i Relacijskih BD}
    \label{tab:comparison}
    \begin{tabular}{|l|l|l|}
    \hline
    \textbf{Kriterij} & \textbf{ZODB} & \textbf{PostgreSQL} \\
    \hline
    Osnovni entitet & Python objekt & Tablica/Redak \\
    \hline
    Upiti & Python kod & SQL \\
    \hline
    Relacije & Direktne reference & Strani ključevi \\
    \hline
    Kompleksnost & Jednostavna & Kompleksna za objekte \\
    \hline
    Tipski sustav & Python tipovi & SQL tipovi \\
    \hline
    \end{tabular}
\end{table}

\section{ACID Svojstva}

ACID predstavlja četiri svojstva koja osiguravaju pouzdanost transakcija u bazama podataka \cite{ACID2020}:
\begin{itemize}
    \item \textbf{Atomarnost (Atomicity)}: Transakcija se ili kompletan izvršava ili se uopće ne izvršava.
    \item \textbf{Konzistentnost (Consistency)}: Baza je uvijek u konzistentnom stanju.
    \item \textbf{Izolacija (Isolation)}: Konkurentne transakcije ne vide međusobne promjene dok se ne commitaju.
    \item \textbf{Trajnost (Durability)}: Kada je transakcija commitana, sprema se trajno.
\end{itemize}

\section{ZODB Specifičnosti}

Objekti koji se mogu pohraniti trebaju naslijediti klasu \texttt{Persistent}. Transakcije se koriste za grupiranje više operacija. Za spravljanje kolekcija u \ac{ZODB}-u koristi se \texttt{PersistentMapping} i \texttt{PersistentList} umjesto običnih Python rječnika i listi \cite{ZODB}.

\chapter{Model Baze Podataka}

Struktura podataka u igri sastoji se od glavnog korijenskog objekta (root) koji sadrži mape za igrače, tablicu najboljih rezultata i stanje svijeta.

\section{Klase i Svojstva}

\subsection{Player Klasa}

Igrač je glavni lik kojega kontrolira korisnik.

\begin{lstlisting}[caption={Definicija Player klase},label={lst:player_class}]
class Player(Persistent):
    def __init__(self, name):
        self.name = name
        self._hp = 100
        self.x = 400
        self.y = 300
        self.inventory = PersistentList()
        self.status = "Aktivan"
\end{lstlisting}

\subsection{Item Klasa}

Stavka je predmet koji igrač može nositi u inventaru.

\begin{lstlisting}[caption={Definicija Item klase},label={lst:item_class}]
class Item(Persistent):
    def __init__(self, name, power):
        self.name = name
        self.power = power
\end{lstlisting}

\chapter{Implementacija}

\section{Struktura Projekta}

Projekt se sastoji od sljedećih datoteka:
\begin{itemize}
    \item \texttt{src/main.py}: Glavna petlja igre i PyGame logika.
    \item \texttt{src/models.py}: Definicije perzistentnih objekata (\texttt{Player}, \texttt{Item}).
    \item \texttt{src/database.py}: Upravljanje \ac{ZODB} vezom i inicijalizacija baze.
    \item \texttt{data/}: Mapa u kojoj se pohranjuju datoteke baze podataka.
\end{itemize}

\section{Inicijalizacija ZODB-a}

Baza podataka se inicijalizira kreiranjem \texttt{FileStorage} i \texttt{DB} objekata, te otvaranjem konekcije \cite{ZODB}.

\begin{lstlisting}[caption={Inicijalizacija baze u database.py},label={lst:db_init}]
class GameDB:
    def __init__(self, db_path='data/game.fs'):
        os.makedirs(os.path.dirname(db_path), exist_ok=True)
        self.storage = ZODB.FileStorage.FileStorage(db_path)
        self.db = ZODB.DB(self.storage)
        self.connection = self.db.open()
        self.root = self.connection.root()
        
        if 'players' not in self.root:
            self.root['players'] = PersistentMapping()
\end{lstlisting}

\section{Okidači (Triggers)}

Okidač za ``HP = 0'' je implementiran kao property setter. Kada se HP promijeni, provjerava se je li pao na 0. Ako jest, status igrača se automatski mijenja.

\begin{lstlisting}[caption={Okidač - property setter za HP},label={lst:trigger}]
@property
def hp(self):
    return self._hp

@hp.setter
def hp(self, value):
    self._hp = max(0, value)
    if self._hp == 0:
        self.status = "Poražen"  # Automatska promjena
    self._p_changed = True  # Javi ZODB-u
\end{lstlisting}

\chapter{Primjeri Korištenja}

\section{Pokretanje i Igranje}

Igra se pokreće naredbom \texttt{python src/main.py}. Igrač se kreće tipkama A i D. Pozicija igrača se ažurira na ekranu u realnom vremenu \cite{Pygame}.

\section{Spremanje Stanja}

Svaki put kada se igra zatvori, poziva se \texttt{transaction.commit()}, čime se svi podaci trajno spremaju u \texttt{data/game.fs}. Pri sljedećem pokretanju, podaci se učitavaju točno onako kako su ostavljeni.

\chapter{Zaključak}

\ac{ZODB} se pokazao odličnim izborom za ovu aplikaciju zbog jednostavnosti direktnog spremanja Python objekata bez mapiranja i fleksibilnosti koju pruža. Property setter-i omogućavaju implementaciju logike okidača na prirodan način. 

Projekat je uspješno demonstrirao kako \ac{ZODB} pohranjuje kompleksne Python objekte, korištenje transakcija za atomske operacije, te integraciju baze s grafičkim sučeljem.

\makebackmatter

\end{document}
