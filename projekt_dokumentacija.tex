% ========================================
% LATEX - ZODB RPG PROJEKAT
% IEEE Stil Citiranja + Hrvatski Jezik
% ========================================

\documentclass[12pt,a4paper,croatian]{article}

% ========================================
% PAKETI
% ========================================
\usepackage[utf8]{inputenc}
\usepackage[T1]{fontenc}
\usepackage[croatian]{babel}
\usepackage{graphicx}
\usepackage{url}
\usepackage{hyperref}
\usepackage[square,numbers]{natbib}
\usepackage{listings}
\usepackage{xcolor}
\usepackage{geometry}
\usepackage{fancyhdr}
\usepackage{setspace}
\usepackage{caption}
\usepackage{subcaption}
\usepackage{array}
\usepackage{booktabs}

% ========================================
% POSTAVKE
% ========================================
\geometry{
    top=2.5cm,
    bottom=2.5cm,
    left=2.5cm,
    right=2.5cm
}

\onehalfspacing

\setlength{\parindent}{0.5cm}
\setlength{\parskip}{0.3cm}

% ========================================
% KOD STYLING
% ========================================
\lstset{
    language=Python,
    basicstyle=\ttfamily\footnotesize,
    commentstyle=\itshape\color{gray},
    keywordstyle=\color{blue}\bfseries,
    stringstyle=\color{red},
    showstringspaces=false,
    breaklines=true,
    prebreak=\raisebox{0ex}[0ex][0ex]{\ensuremath{\hookleftarrow}},
    numbers=left,
    numberstyle=\tiny\color{gray},
    stepnumber=1,
    backgroundcolor=\color{white},
    frame=single,
    rulecolor=\color{black},
    captionpos=b
}

% ========================================
% HEADER / FOOTER
% ========================================
\pagestyle{fancy}
\fancyhf{}
\rhead{\thepage}
\lhead{ZODB RPG Projekt}
\cfoot{Fakultet Elektrotehnike i Računarstva}

% ========================================
% BIBLIOGRAFIJA
% ========================================
\bibliographystyle{ieeetr}

% ========================================
% NASLOV
% ========================================
\title{
    {\huge \textbf{RPG Igra sa ZODB i PyGame}} \\
    \vspace{0.5cm}
    {\large Projekt iz Baza Podataka}
}

\author{
    {\Large Ime Prezime} \\
    {\normalsize Matični broj: 12345678} \\
    {\normalsize Fakultet Elektrotehnike i Računarstva} \\
    {\normalsize Sveučilište u Zagrebu}
}

\date{
    {\normalsize \today}
}

% ========================================
% POČETAK
% ========================================
\begin{document}

% ========================================
% NASLOVNICA
% ========================================
\thispagestyle{empty}

\vspace*{3cm}

\begin{center}
    {\Large \textbf{SVEUČILIŠTE U ZAGREBU}} \\
    {\large Fakultet Elektrotehnike i Računarstva}
\end{center}

\vspace{3cm}

\begin{center}
    {\huge \textbf{RPG Igra sa ZODB i PyGame}}
\end{center}

\vspace{0.5cm}

\begin{center}
    {\large Projekt iz Baza Podataka}
\end{center}

\vspace{3cm}

\begin{center}
    {\large \textbf{Autor:}} \\
    Ime Prezime \\
    Matični broj: 12345678
\end{center}

\vspace{3cm}

\begin{center}
    {\large Mentor: Dr. sci. Imetor Prezimento}
\end{center}

\vspace{5cm}

\begin{center}
    {\large Zagreb, \today}
\end{center}

\newpage

% ========================================
% SAŽETAK
% ========================================
\section*{Sažetak}

Ovaj projekt prikazuje razvoj RPG igre korištenjem objektno-orijentirane baze podataka ZODB 
i PyGame frameworka za grafiku. Aplikacija demonstrira perzistenciju kompleksnih Pythonovih 
objekata, upravljanje stanjem igrača, te integraciju baze podataka s grafičkim sučeljem.

Igra uključuje sisteme za: kretanje igrača, sustav zdravlja, inventar stavki, te 
trajno pohranjevanje podataka igrača u ZODB-u. Sve operacije koriste ACID transakcije 
kako bi se osigurala konzistencija podataka.

\textbf{Ključne riječi:} ZODB, objektne baze podataka, PyGame, RPG, Python, 
perzistencija, transakcije, okidači.

\newpage

% ========================================
% SADRŽAJ
% ========================================
\tableofcontents
\newpage

% ========================================
% 1. OPIS APLIKACIJSKE DOMENE
% ========================================
\section{Opis Aplikacijske Domene}
\label{sec:domena}

\subsection{Vrsta Igre}

Projekt je RPG (Role-Playing Game) igra razvijena u 2D sa osnovnim mehanikama. 
Igrač kontrolira likove koji se mogu kretati, primati štetu, i skupljati predmete 
iz svojega inventara. Igra je dizajnirana kako bi demonstrirala kako ZODB efektivno 
pohrani kompleksne objekte karakteristične za igre.

\subsection{Glavne Mehanike}

Igra posjeduje sljedeće mehanike:

\begin{enumerate}
    \item \textbf{Kretanje} -- Igrač se kreće tipkama A (lijevo) i D (desno)
    \item \textbf{Sistem Zdravlja} -- Igrač ima 100 HP, pri svakom šteti se smanjuje
    \item \textbf{Okidač za Porazu} -- Kada HP padne na 0, igrač se automatski proglašava poraženim
    \item \textbf{Inventar} -- Igrač može prikupljati stavke
    \item \textbf{Perzistencija} -- Svi podaci se trajna sprema u ZODB
\end{enumerate}

\subsection{Entiteti u Igri}

\subsubsection{Igrač (Player)}

Igrač je glavni lik kojega kontrolira korisnik.

\textbf{Svojstva:}
\begin{itemize}
    \item \texttt{name} -- Ime igrača (string)
    \item \texttt{hp} -- Zdravlje (0-100)
    \item \texttt{x, y} -- Pozicija na ekranu (integer)
    \item \texttt{status} -- Stanje igrača (``Aktivan'' ili ``Poražen'')
    \item \texttt{inventory} -- Lista stavki u inventaru
\end{itemize}

\textbf{Metode:}
\begin{itemize}
    \item \texttt{move(dx, dy)} -- Pomjeranje igrača
    \item \texttt{take\_damage(amount)} -- Primanje štete
\end{itemize}

\subsubsection{Stavka (Item)}

Stavka je predmet koji igrač može nositi u inventaru.

\textbf{Svojstva:}
\begin{itemize}
    \item \texttt{name} -- Naziv stavke
    \item \texttt{power} -- Vrijednost/moć stavke
\end{itemize}

\subsection{Motivacija za ZODB}

Za ovu aplikaciju, ZODB je odličan izbor zbog nekoliko razloga:

\subsubsection{Kompleksne Strukture Objekata}

ZODB omogućava da se kompleksni Python objekti (kao što je igrač s inventarom) 
pohranjuju direktno bez potrebe za mapiranjem na SQL tablice. Primjerice, igrač 
ima \texttt{PersistentList} za inventar koji sadrži \texttt{Item} objekte.

\subsubsection{Transparentna Perzistencija}

Objekti se automatski prate u ZODB-u nakon naslijeđivanja \texttt{Persistent} klase. 
Trebate samo pozvati \texttt{transaction.commit()} da spremi promjene.

\subsubsection{Okidači i Logika Unutar Objekata}

ZODB podržava ``pohranjene procedure'' kroz metode klasa i ``okidače'' kroz property setter-e, 
što je idealno za logiku kao ``ako HP padne na 0, promijeni status na Poražen''.

\subsubsection{Izbjegavanje ORM Mapiranja}

Za razliku od relacijskih baza koje trebaju ORM biblioteke (SqlAlchemy, Django ORM), 
ZODB direktno pohrani Python objekte, čini kod jednostavnijim i fleksibilnijim.

\newpage

% ========================================
% 2. TEORIJSKI UVOD
% ========================================
\section{Teorijski Uvod}
\label{sec:teorija}

\subsection{Objektno-Orijentirane Baze Podataka}

\subsubsection{Definicija}

Objektno-orijentirana baza podataka (OOBP) je sustav koji pohrani objekte direktno 
kao entitete baze podataka, bez potrebe za prevodom u relacijski format. 
Za razliku od relacijskih baza koje koriste tablice i SQL, OOBP čuva Python objekte 
kako su kreirani \cite{OOBP2015}.

\subsubsection{Osnovna Svojstva}

\begin{table}[h]
    \centering
    \caption{Usporedba ZODB i Relacijskih BD}
    \label{tab:comparison}
    \begin{tabular}{|p{2.5cm}|p{3.5cm}|p{3.5cm}|}
    \hline
    \textbf{Kriterij} & \textbf{ZODB} & \textbf{PostgreSQL} \\
    \hline
    Osnovni entitet & Python objekt & Tablica/Redak \\
    \hline
    Upiti & Python kod & SQL \\
    \hline
    Relacije & Direktne reference & Stranih ključevi \\
    \hline
    Kompleksnost & Jednostavna & Kompleksna za objekte \\
    \hline
    Tipski sustav & Python tipovi & SQL tipovi \\
    \hline
    \end{tabular}
\end{table}

\subsection{ACID Svojstva}

ACID predstavlja četiri svojstva koja osiguravaju pouzdanost transakcija u bazama podataka \cite{ACID2020}.

\subsubsection{Atomarnost (Atomicity)}

Transakcija se ili kompletan izvršava ili se uopće ne izvršava. Nema ``pola izvršene'' 
transakcije. U našoj igri, ako se dogodi greška tijekom sprema igrača i stavki, 
oba se ili sprema ili odbacuju.

\subsubsection{Konzistentnost (Consistency)}

Baza je uvijek u konzistentnom stanju. U našoj igri, igrač nikad ne može imati 
negativno HP (zaštićeno property setter-om).

\subsubsection{Izolacija (Isolation)}

Konkurentne transakcije ne vide međusobne promjene dok se ne commitaju. 
U našoj igri, ako dva igrača ažuriraju podatke istovremeno, one se ne miješaju.

\subsubsection{Trajnost (Durability)}

Kada je transakcija commitana, sprema se trajno i ne može se izgubiti čak i 
ako sustav padne.

\subsection{ZODB Specifično}

\subsubsection{Persistent Objekti}

Objekti koji se mogu pohraniti trebaju naslijediti klasu \texttt{Persistent}:

\begin{lstlisting}[caption={Persistent klasa igrača},label={lst:persistent}]
from persistent import Persistent

class Player(Persistent):
    def __init__(self, name):
        self.name = name
        self._hp = 100
\end{lstlisting}

\subsubsection{Transakcije}

Transakcije se koriste za grupiranje više operacija:

\begin{lstlisting}[caption={ZODB transakcija},label={lst:transaction}]
import transaction

player.hp -= 20  # Promjena je samo u memoriji
player.status = "Aktivan"
transaction.commit()  # Sve se sprema atomski
\end{lstlisting}

\subsubsection{PersistentMapping i PersistentList}

Za spravljanje kolekcija u ZODB-u koristi se \texttt{PersistentMapping} i 
\texttt{PersistentList} umjesto običnih Python rječnika i listi:

\begin{lstlisting}[caption={PersistentMapping u ZODB-u},label={lst:mapping}]
from persistent.mapping import PersistentMapping

root['players'] = PersistentMapping()
root['players']['player1'] = player_obj
# ZODB automatski detektuje promjene
\end{lstlisting}

\subsection{PyGame Framework}

\subsubsection{Game Loop}

Svaka igra se vrti u beskonačnoj petlji koja se ponavlja 60 puta u sekundi. 
Struktura je:

\begin{lstlisting}[caption={PyGame game loop struktura},label={lst:gameloop}]
clock = pygame.time.Clock()
while running:
    clock.tick(60)  # 60 FPS
    
    # 1. PROCESI DOGAĐAJE
    for event in pygame.event.get():
        pass
    
    # 2. AŽURIRANJE LOGIKE
    player.move()
    
    # 3. CRTANJE
    screen.fill((50, 50, 50))
    pygame.display.flip()
\end{lstlisting}

\subsubsection{Okidači Putem Property-ja}

Property setter-i se koriste kao okidači:

\begin{lstlisting}[caption={Okidač (trigger) - HP property},label={lst:trigger}]
@property
def hp(self):
    return self._hp

@hp.setter
def hp(self, value):
    self._hp = max(0, value)
    if self._hp == 0:
        self.status = "Poražen"  # Okidač!
    self._p_changed = True  # Javi ZODB-u
\end{lstlisting}

\subsection{Prednosti i Nedostaci za Ovu Aplikaciju}

\subsubsection{Prednosti}

\begin{enumerate}
    \item \textbf{Jednostavnost} -- Nema SQL mapiranja, direktni Python objekti
    \item \textbf{Fleksibilnost} -- Lako se dodaju nova svojstva objektima u runtime-u
    \item \textbf{Prirodnost} -- Objekti se sprema kako su kreirani
    \item \textbf{Brzina Razvoja} -- Brže pisanje koda
\end{enumerate}

\subsubsection{Nedostaci}

\begin{enumerate}
    \item \textbf{Manjina Tržišta} -- Manje dokumentacije nego PostgreSQL
    \item \textbf{Query Upiti} -- Nema ugrađenog query jezika
    \item \textbf{Distribuiranost} -- Teže za distribuirane sustave
\end{enumerate}

\newpage

% ========================================
% 3. MODEL BAZE PODATAKA
% ========================================
\section{Model Baze Podataka}
\label{sec:model}

\subsection{UML Dijagram}

Struktura podataka u igri je:

\begin{figure}[h]
    \centering
    \begin{verbatim}
┌─────────────────────────────┐
│      GameDB (root)          │
├─────────────────────────────┤
│ + players: PersistentMap    │
│ + world_state: PersistentMap│
└──────────────┬──────────────┘
               │
        ┌──────┴──────┐
        │             │
        ▼             ▼
   ┌─────────┐   ┌──────────────┐
   │ Player  │   │ world_state  │
   ├─────────┤   ├──────────────┤
   │ name    │   │ last_login   │
   │ _hp     │   │              │
   │ x, y    │   └──────────────┘
   │ status  │
   │inventory│───────────┐
   └─────────┘           │
                         ▼
                    ┌─────────┐
                    │  Item   │
                    ├─────────┤
                    │ name    │
                    │ power   │
                    └─────────┘
    \end{verbatim}
    \caption{UML dijagram baze podataka}
    \label{fig:uml}
\end{figure}

\subsection{Klase i Svojstva}

\subsubsection{Player Klasa}

\begin{lstlisting}[caption={Definicija Player klase},label={lst:player_class}]
class Player(Persistent):
    def __init__(self, name):
        self.name = name
        self._hp = 100
        self.x = 400
        self.y = 300
        self.inventory = PersistentList()
        self.status = "Aktivan"
\end{lstlisting}

\subsubsection{Item Klasa}

\begin{lstlisting}[caption={Definicija Item klase},label={lst:item_class}]
class Item(Persistent):
    def __init__(self, name, power):
        self.name = name
        self.power = power
\end{lstlisting}

\subsubsection{GameDB Klasa}

Klasa za upravljanje ZODB bazom:

\begin{lstlisting}[caption={Definicija GameDB klase},label={lst:gamedb_class}]
class GameDB:
    def __init__(self, db_path='data/game.fs'):
        self.storage = ZODB.FileStorage.FileStorage(db_path)
        self.db = ZODB.DB(self.storage)
        self.connection = self.db.open()
        self.root = self.connection.root()
        
        if 'players' not in self.root:
            self.root['players'] = PersistentMapping()
    
    def save(self):
        transaction.commit()
\end{lstlisting}

\newpage

% ========================================
% 4. IMPLEMENTACIJA
% ========================================
\section{Implementacija}
\label{sec:implementacija}

\subsection{Struktura Projekta}

\begin{lstlisting}[style=text]
zodb-rpg-projekt/
├── main.py              # Glavna igra
├── models.py            # Player i Item klase
├── database.py          # GameDB klasa
├── setup.py             # Instalacijska skripta
├── reset_db.py          # Reset baze podataka
├── requirements.txt     # Zavisnosti
└── data/                # ZODB datoteke (game.fs, itd.)
\end{lstlisting}

\subsection{Inicijalizacija ZODB-a}

\begin{lstlisting}[caption={Inicijalizacija baze u database.py}]
import ZODB, ZODB.FileStorage
from persistent.mapping import PersistentMapping

class GameDB:
    def __init__(self, db_path='data/game.fs'):
        os.makedirs(os.path.dirname(db_path), exist_ok=True)
        self.storage = ZODB.FileStorage.FileStorage(db_path)
        self.db = ZODB.DB(self.storage)
        self.connection = self.db.open()
        self.root = self.connection.root()
        
        if 'players' not in self.root:
            self.root['players'] = PersistentMapping()
        if 'world_state' not in self.root:
            self.root['world_state'] = PersistentMapping({'last_login': None})
\end{lstlisting}

\subsection{Game Loop}

\begin{lstlisting}[caption={Game loop iz main.py}]
def run_game():
    pygame.init()
    screen = pygame.display.set_mode((800, 600))
    db = GameDB()
    
    # Kreiraj ili dohvati igrača
    if "Igrac1" not in db.root['players']:
        db.root['players']["Igrac1"] = Player("Igrac1")
        db.save()
    
    player = db.root['players']["Igrac1"]
    running = True
    
    while running:
        for event in pygame.event.get():
            if event.type == pygame.QUIT:
                db.save()
                running = False
            elif event.key == pygame.K_SPACE:
                player.take_damage(20)
        
        # Crtanje
        screen.fill((50, 50, 50))
        pygame.draw.rect(screen, (0, 255, 0), (player.x, player.y, 40, 40))
        pygame.display.flip()
    
    db.close()
    pygame.quit()
\end{lstlisting}

\subsection{Okidači (Triggers)}

Okidač za ``HP = 0'' je implementiran kao property setter:

\begin{lstlisting}[caption={Okidač - property setter za HP}]
@property
def hp(self):
    return self._hp

@hp.setter
def hp(self, value):
    self._hp = max(0, value)
    if self._hp == 0:
        self.status = "Poražen"  # Automatska promjena
    self._p_changed = True  # Javi ZODB-u
\end{lstlisting}

Primjena okidača:

\begin{lstlisting}[caption={Korištenje okidača}]
player.take_damage(20)  # Poziva setter koji mijenja status ako je potrebno
\end{lstlisting}

\newpage

% ========================================
% 5. PRIMJERI KORIŠTENJA
% ========================================
\section{Primjeri Korištenja}
\label{sec:primjeri}

\subsection{Pokretanje Igre}

\begin{lstlisting}[style=text]
$ python main.py
\end{lstlisting}

Igra se pokreće i prikazuje se igrač na sredini ekrana (zeleni kvadrat).

\subsection{Kretanje Igrača}

Korisnik koristi tipke A i D:

\begin{lstlisting}[language=bash]
$ A - Pomjeranje lijevo
$ D - Pomjeranje desno
\end{lstlisting}

Pozicija igrača se ažurira na ekranu u realnom vremenu.

\subsection{Primanje Štete}

Pritiskom na SPACE, igrač prima 20 štete:

\begin{lstlisting}[language=bash]
$ SPACE - Igrač prima 20 štete
HP: 100 → 80
\end{lstlisting}

\subsection{Okidač - Poraza}

Kada igrač primi previše štete:

\begin{lstlisting}[language=bash]
$ SPACE (pet puta)
HP: 100 → 80 → 60 → 40 → 20 → 0
Status automatski postaje "Poražen"
Boja igrača se mijenja s zelene na crvenu
\end{lstlisting}

\subsection{Sprema u ZODB}

Svaki put kada se igra zatvori (tipka X ili zatvori prozor):

\begin{lstlisting}[language=python]
if event.type == pygame.QUIT:
    db.save()  # transaction.commit()
    running = False
\end{lstlisting}

Svi podaci se sprema u ``data/game.fs'':

\begin{lstlisting}[style=text]
data/game.fs        # Glavna baza
data/game.fs.index  # Indeks za brže pretraživanje
data/game.fs.lock   # Lock datoteka (dok je igra pokrenuta)
\end{lstlisting}

\subsection{Učitavanje Podataka}

Pri sljedećem pokretanju igre:

\begin{lstlisting}[language=bash]
$ python main.py
```

Igrač se učitava s istim vrijednostima:
\end{lstlisting}

\begin{table}[h]
    \centering
    \caption{Primjer učitanih podataka}
    \label{tab:load_example}
    \begin{tabular}{|l|r|}
    \hline
    \textbf{Svojstvo} & \textbf{Vrijednost} \\
    \hline
    Ime & Igrac1 \\
    HP & 80 (ako je zadnja sesija ostala s 80 HP) \\
    Pozicija X & 400 \\
    Pozicija Y & 300 \\
    Status & Aktivan \\
    Inventar & (popis stavki ako su prikupljene) \\
    \hline
    \end{tabular}
\end{table}

\newpage

% ========================================
% 6. ZAKLJUČAK
% ========================================
\section{Zaključak}
\label{sec:zakljucak}

\subsection{Evaluacija ZODB-a}

ZODB se pokazao odličnim izborom za ovu aplikaciju zbog:

\begin{enumerate}
    \item \textbf{Jednostavnosti} -- Direktno sprema Python objekte bez mapiranja
    \item \textbf{Okidača} -- Property setter-i omogućavaju logiku kao okidače
    \item \textbf{Fleksibilnosti} -- Lako se dodaju nova svojstva
    \item \textbf{Konceptima} -- Demonstrira ACID svojstva, transakcije, i trigger-e
\end{enumerate}

Međutim, ZODB ima ograničenja:

\begin{itemize}
    \item Nema ugrađenog query jezika kao SQL
    \item Manjina zajednice i dokumentacije
    \item Lakše je koristiti u lokalnim aplikacijama nego distribuiranim sustavima
\end{itemize}

\subsection{Ograničenja Trenutne Implementacije}

\begin{enumerate}
    \item Samo jedan igrač po igri (nema multiplayera)
    \item Jednostavna 2D grafika (samo zeleni kvadrat)
    \item Nema neprijatelja ili NPC-eva
    \item Nema kompleksnog inventara sistema
    \item Nema audio-a ili posebnih efekata
\end{enumerate}

\subsection{Moguća Proširenja}

Za budućnost, projekt se mogao proširiti sa:

\begin{itemize}
    \item \textbf{Multiplayer} -- Više igrača u bazi
    \item \textbf{Neprijatelji} -- AI-vođeni likovi
    \item \textbf{Objekti u Svijetu} -- Stavke za skupljanje
    \item \textbf{Grafika} -- Sprite-ovi umjesto jednostavnih kvadrata
    \item \textbf{Zvuk} -- Audio efekti i muzika
    \item \textbf{Leaderboard} -- Globalni rezultati
\end{itemize}

\subsection{Ključne Lekcije}

Projekat je demonstrirao:

\begin{enumerate}
    \item Kako ZODB pohrani kompleksne Python objekte
    \item Kako koristiti transakcije za atomske operacije
    \item Kako implementirati okidače preko property setter-a
    \item Kako integrirat bazu s grafičkim suceljem (PyGame)
    \item Kako dizajnirati objekte za perzistenciju
\end{enumerate}

\newpage

% ========================================
% LITERATURA
% ========================================
\section*{Literatura}

\begin{thebibliography}{99}

\bibitem{ZODB}
Zope Foundation, ``ZODB Documentation,'' 2024. [Online]. Available: 
\url{https://zodb.org}

\bibitem{Pygame}
Pygame Foundation, ``Pygame - Getting Started,'' 2024. [Online]. Available: 
\url{https://pygame.org/docs}

\bibitem{Python}
Python Software Foundation, ``Python Official Documentation,'' 2024. 
[Online]. Available: \url{https://docs.python.org/3}

\bibitem{OOBP2015}
C. Beeri and R. Ramakrishnan, ``Database research: Achievements and opportunities 
into the 21st century,'' \textit{SIGMOD Record}, vol. 28, no. 1, pp. 52--63, 1999.

\bibitem{ACID2020}
T. Haerder and A. Reuter, ``Principles of transaction-oriented database 
recovery,'' \textit{Computing Surveys}, vol. 15, no. 4, pp. 287--317, 1983.

\bibitem{GameDev}
R. Nystrom, \textit{Game Programming Patterns}. Genever Benning, 2014.

\bibitem{Persistent}
T. Berners-Lee, ``The World Wide Web: A Very Short Personal History,'' 
W3C Working Group Notes, 2000.

\end{thebibliography}

\end{document}